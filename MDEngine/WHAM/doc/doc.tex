\documentclass[12pt]{article}
\usepackage{url}
\usepackage[pdftex,colorlinks=true,urlcolor=blue,filecolor=blue,linkcolor=blue,citecolor=blue,breaklinks=true]{hyperref}
\begin{document}

\begin{center}
\begin{LARGE}
{\bf An implementation of WHAM: the Weighted Histogram Analysis Method} \\
{\bf Version 2.0.9} \\
\vspace*{0.5in}
{\bf Alan Grossfield} \\
\end{LARGE}
\end{center}

\tableofcontents

\newpage

\section{Introduction}

These programs (wham and wham-2d) implement the Weighted Histogram
Analysis Method of Kumar, et al (``Multidimensional free-energy
calculations using the weighted histogram analysis method'',
J. Comput. Chem., 16:1339-1350, 1995).  The code generally
follows the notation used by Benoit Roux (``The calculation of the
potential of mean force using computer simulations'', Comput. Phys. Comm.,
91:275-282, 1995).  Consult these papers for the theoretical background
and justification for the method.

This code is available for download from my web page
(\url{http://membrane.urmc.rochester.edu/content/wham/}). The code doesn't
change all that often, but it's probably worth checking periodically.  If you
run into trouble using these programs, feel free to contact me
(alan\_grossfield@urmc.rochester.edu), and I'll try to help you.  This code
is available under the GPL and BSD licenses, as you prefer.  The exception to
this licensing is a set of routines from Numerical Recipes, which are not mine
to give away.  

If you use this code as part of an original piece of research, I'd
appreciate a reference or acknowledgment.  There's no publication to
reference, so please use something like:

\noindent
Grossfield, A, ``WHAM: an implementation of the weighted histogram analysis 
method'', \url{http://membrane.urmc.rochester.edu/content/wham/}, version
XXXX

\noindent For that matter, just letting me know what you're using my code for
would be nice, although again I don't insist upon it.

Suggestions and patches are welcome.  

\section{New in release 2.0.9}

The is a bug-fix release; there was a memory allocation error in wham-2d
that caused a segfault if the number of bins in the 2nd dimension was
greater than the number of windows.  Caught by Bastien Loubet from the
Center for Biomembrane Physics in Denmark.

\section{New in release 2.0.8}

Major performance improvements to wham-2d (roughly 100x faster).  The final
answers should be indistinguishable from previous versions.  Code
contributed by Nikolay Plotnikov from Stanford's Chemistry Department.

\section{New in release 2.0.7}

I messed up the ability to switch units from kcal to kJ: in addition to
defining k\_B in the header files, I also had a define in wham\_2d.c.  Caught
by Yi Yao of UNC.

\section{New in release 2.0.6}

The primary new feature in version 2.0.6 is the ability to exclude regions
of a 2D reaction coordinate from a calculation, which we're calling
``masking''.  This is useful in cases where the PMF is being used to sample a
pathway that is two dimensional, but where large areas of the 2D surface are
either uninteresting or physically unattainable; for example, if the two
reaction coordinates are RMSDs from 2 distinct structures, it is almost
certainly impossible to sample the region where the RMS distance to each
structure is very small.  In practice, you could generally do the calculation
anyway and just accept lots of NaNs and Infs floating around in the output,
but sometimes that would fail and this way is far cleaner and robust.  In
this case, I chose to implement a very simple automasking procedure, such
that any bin that has no data from any window is excluded.  I suppose I could
further generalize it, so that any bin with fewer than N points is excluded,
but I haven't done that yet.

There's one other trivial but often requested feature in version 2.0.6: if
you prefer to work in SI units, I've made it easier to switch energy units 
from kcal to kJ (the units of your reaction coordinate are, as always, up to
you).  It's a compile-time choice, but I figure that's fine because very few
people switch units constantly.  The switch is easy -- before you build the
code, edit {\tt wham/wham.h} and {\tt wham-2d/wham-2d.h} and change which
version of the constant {\tt k\_B} is defined.

\section{New in release 2.0.4 and 2.0.5}

Version 2.0.5 is a trivial patch on 2.0.4, upping the length of lines allowed
when reading files.

Release 2.0.4 has seen major revision to the bootstrap error analysis.  
I've known for a while that the way I computed the uncertainty in the free
energy was suboptimal, since it just extrapolated from the uncertainty in the
probability using the assumption that the fluctuations were gaussianly
distributed.  I knew this wasn't a great assumption, but Michael Shirts
showed me data showing just how bad it was.  So, in this version, we're doing
something different.  For 1D wham, we're directly computing the fluctuations
in the PMF at each bin, aligning the pmfs such that their partition functions
are all 1; this amounts to shifting them so that their Boltzmann-weighted
free energies are the same.  I think the errors make more sense now.

For 2D wham, I think doing this analysis uncovered some other flaws in how the
error analysis is done, ones that to be honest I'm not totally sure how to
fix.  For now, I'm simply removing the option to do bootstrapping in 2D.  I
hope to put it back once I solve the problem, in which case I'll make another
release.

Many thanks to Michael Shirts, for helpful discussions and contributing much
of the code changes that went into this release.

\section{Installation}

Untarring wham.tgz will create a directory wham/, which in turn contains
several directories (wham/ wham-2d/ doc/ nr/).   

Before you build the code, you need to choose your units for energy.  If you
want kcal/mol, do nothing.  If you prefer to work in kJ, you need to edit
2 files, {\tt wham/wham.h} and {\tt wham-2d/wham-2d.h}.  In each case, near the
top of the file you'll see a pair of lines that look like 

\begin{verbatim}
#define k_B 0.001982923700 // Boltzmann's constant in kcal/mol K
//#define k_B  0.0083144621 // Boltzmann's constant kJ/mol-K
\end{verbatim}

To get kJ, you simply remove the ``//'' from the second line and put it
before the first line, and save the file.  In principle, you can do just wham
or just wham-2d, but I recommend against it for your own sanity.  

To build the standard 1-D wham code on a machine which has gcc, you should

\begin{verbatim}
    cd wham
    make clean
    make
\end{verbatim}

(which will delete all object files and executables, and build the wham
executable).  By default, the Makefile uses the gcc compiler, but I also have
flags present for the Intel compiler, plus the native compilers for Irix and
Tru64.  The latter two have not been tested recently, but ought to work,
since at various times this code has been used successfully on various
flavors of linux, MacOS X, AIX, Irix, and Tru64.  If you find that you need
to do anything special to make it work on your particular system, I'd
appreciate it if you could let me know so I can add to the Makefile.

To build the 2-D version,  say

\begin{verbatim}
    cd wham-2d
    make clean
    make
\end{verbatim}

Several other directories are also created (doc/, which you presumably found
because you're reading this, and nr/, which contains a couple of files from
Numerical Recipes).  You don't need to do anything with these directories.

\section{Command line arguments and file formats}

To get a listing of the command line arguments for either wham or wham-2d,
just run the command without any arguments.  Optional arguments are
included in brackets.  Both programs will echo their command line into the
output file, to help you figure out what you did.

\subsection{wham}

\subsubsection{Command line arguments for WHAM}

\begin{verbatim}
wham [P|Ppi|Pval] hist_min hist_max num_bins tol temperature numpad  \
        metadatafile freefile [num_MC_trials randSeed]
\end{verbatim}

The first (optional) argument specifies the periodicity of the reaction
coordinate.  For a nonperiodic reaction coordinate (a distance, for
example), it should be left out.  ``P'' means that the reaction coordinate
has a periodicity of 360, appropriate for angles. ``Ppi'' specifies a
periodicity of 2*pi, appropriate for angles measured in radians.  ``Pval''
specifies periodicty of some arbitrary amount, val, which should be an
integer or floating point number.  For example, ``P180.0'' would be
appropriate for an angle with twofold symmetry. 

hist\_min and hist\_max specify the boundaries of the histogram.  As a rule,
all data points outside the range (hist\_min, hist\_max) are silently
ignored.  The only exception is that if an entire trajectory is outside the
range, the program halts with an error message.  The solution is to remove
that file from the metadata file.  hist\_min and hist\_max should be floating
point numbers.

num\_bins specifies the number of bins in the histogram, and as a result the
number of points in the final PMF.  It should be an integer.

tol is the convergence tolerance for the WHAM calculations.  Specifically,
the WHAM iteration is considered to be converged when no $F_i$ value for any
simulation window changes by more than tol on consecutive iterations.  As
the program runs, it prints the average change in the F values for the most
recent iteration.  Obviously, this number will be smaller than tol
before the computation converges, because convergence is triggered by the
largest change as opposed to the average.

temperature is a floating point number representing the temperature in
Kelvin at which the weighted histogram calculation is performed.  This does
not have to be the temperature at which the simulations were performed (see
below for discussion).

numpad specifies the number of ``padding'' values that should be printed for
periodic PMFs.  This number should be set to 0 for aperiodic reaction
coordinates.  It doesn't actually affect the calculation in any way.
Rather, it just alters the final printout of the free energy, to make
plotting of periodic reaction coordinates simpler.  This is more important
for wham-2d than wham.

metadatafile specifies the name of the metadata file.  The format of this
file is described below.

freefile is the name used for the file containing the final PMF and
probability distribution.

num\_MC\_trials and randSeed are both related to the performance of Monte
Carlo bootstrap error analysis.  If these values are not supplied, error
analysis is not performed.  num\_MC\_trials should be an integer specifying
the number of fake data sets which should be generated.  randSeed is an
integer which controls the random number seed -- the value you pick should
be irrelevant, but I let the user set it primarily for debugging purposes.

\subsubsection{File formats}
\label{ss:format}

Each line of the metadata file should be blank, begin with a ``\#'' (marking a 
comment), or have the following format:

\begin{footnotesize}
\begin{verbatim}
/path/to/timeseries/file    loc_win_min   spring   [correl time] [temperature]
\end{verbatim}
\end{footnotesize}

This first field is the name of one of the time series files (more on this
in a moment).  The second field, loc\_win\_min, is the location of the
minimum of the biasing potential for this simulation, a floating point
number.  The third field, spring, is the spring constant for the biasing
potential used in this simulation, assuming the biasing potential is of the
format 

\begin{equation}
V = \frac{1}{2} k (x-x_0)^2.  
\end{equation}

Many simulation packages, including TINKER, AMBER, and CHARMM, do not include
the $\frac{1}{2}$ when they specify spring constants for their restraint
terms.  This is a common source of error (I'd love to change my code to match
the other packages' behavior, but then experienced users who don't read the
manual would get messed up).  Also, the units for the spring constant must
match those for the time series.  So, if your time series is a distance
recorded in {\AA}ngstroms, the spring constant must be in kcal/mol-{\AA}$^2$.
AMBER users should take care when using angular restraints: the specification
and output of angles is in degrees, but AMBER's spring constants use
kcal/mol-rad$^2$.

The fourth argument (``correl time'') specifies the decorrelation time for
your time series, in units of time steps.  It is only used when generating
fake data sets for Monte Carlo bootstrap error analysis, where it modulates
the number of points per fake data set.  This argument is optional, and is
ignored if you don't do error analysis.  If you're doing multiple
temperatures but not bootstrapping, set it to any integer value as a
placeholder, and it'll be ignored.  See section \ref{ss:bootstrap} for more
discussion about how to do bootstrapping.

Finally, the last (optional) field is the temperature for this simulation.  If
not supplied, the temperature specified on the command line is used.  In
the present version of the code, you must either leave the temperature
unspecified for all simulations or specify it for all simulations.  

The time series files must follow one of two formats, depending on whether
the temperature was specified in the metadata file.  If no temperature was
specified, the file should contain two columns, where the first is the time
(which isn't actually used), and the second is the position of the system
along the reaction coordinate.  Both numbers should be in floating point
format.  Lines beginning with ``\#'' are ignored as comments.  Additional
columns of data are ignored.

If the simulation temperature is specified, there must be a third column of
data, containing the system's potential energy at that time point.  It
should be a floating point value.

\subsubsection{Output}

The first line of the output file contains echoes command line.  The next
line or two contain comments describing the periodicity used and the number
of simulation windows present.  
While the calculation is running, it will print out lines that look like
the following:
\begin{verbatim}
#Iteration 10:  0.106019
#Iteration 20:  0.062269
#Iteration 30:  0.039890
#Iteration 40:  0.027003
\end{verbatim}

This specifies the current iteration number, and the average change in the
F values for the current iteration.  This number is not used for deciding
when the calculation has converged; rather, the maximum change, as opposed
to the average, is used.  

Every 100 iterations, the current version of the PMF is dumped into the
output file.  These lines look like

\begin{verbatim}
-178.000000     0.014212        4909.138943
-174.000000     0.062631        4525.390035
-170.000000     0.227076        3432.434076
-166.000000     0.494262        2190.487110
-162.000000     0.817734        1271.708620
\end{verbatim}

The first column is the value of the reaction coordinate, the second is the
value of the PMF, and the third is the unnormalized probability
distribution.

Once the calculation has converged, wham will produce output resembling

\begin{verbatim}
# Dumping simulation biases, in the metadata file order 
# Window  F (free energy units)
# 0     0.000004
# 1     -4.166136
# 2     -3.241052
# 3     -4.475215
# 4     -6.324340
# 5     -7.128731
\end{verbatim}

These are the final F values from the wham calculation, and can be used for
computing weighted averages for properties other than the free energy.

You may have noticed that all of the lines except the free energies are
preceded by ``\#''.  This allows you to check convergence of your wham
calculation by simply plotting the output file in gnuplot.  If the free
energy curves have stopped changing, your tolerance is small enough.  

If you specified a nonzero number of Monte Carlo bootstrap error analysis
trials, you will see lines that resemble

\begin{verbatim}
#MC trial 0: 990 iterations
#MC trial 1: 973 iterations
#MC trial 2: 970 iterations
#MC trial 3: 981 iterations
#MC trial 4: 984 iterations
\end{verbatim}

at the end of the file.

The free energy data file is written when the calculation converges, and
resembles: 

\begin{verbatim}
#Coor           Free            +/-             Prob            +/-
-178.000000     0.014386        0.000098        0.106389        0.000017
-174.000000     0.068560        0.000151        0.097128        0.000025
-170.000000     0.250825        0.000350        0.071496        0.000042
-166.000000     0.523786        0.000294        0.045186        0.000022
\end{verbatim}

The first column is the value of the reaction coordinate, the second is the
free energy.  The third is the statistical uncertainty of the free energy
(which is only meaningful if you performed Monte Carlo bootstrapping).  The
fourth and fifth columns are the probability and it's associated
statistical uncertainty.  Again, the latter is only meaningful if
bootstrapping is performed.  See section \ref{ss:bootstrap} for further
discussion of error estimation.

\subsection{wham-2d}

\subsubsection{Command line arguments}

\begin{verbatim}
wham-2d Px[=0|pi|val] hist_min_x hist_max_x num_bins_x  \
        Py[=0|pi|val] hist_min_y hist_max_y num_bins_y  \
        tol temperature numpad metadatafile freefile \ 
        use_mask
\end{verbatim}

The command line arguments largely have the same meaning as they do for the
one dimensional wham program.

The periodicity arguments are not optional.

``Px'' by itself indicates that the first dimension of the reaction coordinate
has a period of 360.  ``Px=0'' turns off periodicity.  ``Px=pi'' specifies a
period of 2*pi, and ``Px=val'' allows you to choose an arbitrary value for
the period.

hist\_min\_x, hist\_max\_x, and num\_bins\_x behave exactly like hist\_min,
hist\_max, and num\_bins do in the 1 dimensional program.

Py, hist\_min\_y, etc., behave the same as Px, hist\_min\_x, etc., except they
control the second coordinate of the PMF.

use\_mask expects an integer value, and if its values is non-zero turns on
the automasking feature, which causes bins for which there is no sample data
to be excluded from the wham calculation.

\subsubsection{File formats}

As with regular 1 dimensional wham, each line of the metadata file should
either be blank, begin with a ``\#'', or have the following format

\begin{footnotesize}
\begin{verbatim}
/path/to/timeseries/file loc_win_x loc_win_y spring_x spring_y [correl time] [temp]
\end{verbatim}
\end{footnotesize}

This first field is the name of one of the time series files.  loc\_win\_x
and loc\_win\_y are the locations of the minimum of the biasing terms in the
first and second dimensions of the reaction coordinate.  spring\_x and
spring\_y are the spring constants used for the biasing potential in this
simulation, assuming the biasing potential is of the format

\begin{equation}
V = \frac{1}{2} ( k_x (x - x_0)^2 + k_y (y -y_0)^2 )
\end{equation}

The sixth argument (``correl time'') specifies the decorrelation time for
your time series, in units of time steps.  It is only used when generating
fake data sets for Monte Carlo bootstrap error analysis, where it modulates
the number of points per fake data set.  This argument is optional, and is
ignored if you don't do error analysis.  If you're doing multiple
temperatures but not bootstrapping, set it to any integer value as a
placeholder, and it'll be ignored.  See section \ref{ss:bootstrap} for more
discussion about how to do bootstrapping.

Finally, the last field is the temperature this simulation was run at.  If
not supplied, the temperature specified on the command line is used.  In
the present version of the code, you must either leave the temperature
unspecified for all simulations or specify it for all simulations.  

The time series files must follow one of two formats, depending on whether
the temperature was specified in the metadata file.  If no temperature was
specified, the file should contain three columns, where the first is the
time (which isn't actually used), and the second and third are the position
of the system along the x and y reaction coordinates, respectively.  Both
numbers should be in floating point format.  Lines beginning with ``\#'' are
ignored as comments.  Additional columns of data are ignored.

If the simulation temperature is specified, there must be a fourth column of
data, containing the system's potential energy at that time point.  It
should be a floating point value.  See the section on replica exchange for
more details.

\subsubsection{Output}

The output largely resembles that for wham, except with more columns.  The
first line echoes the command line, followed by a specification of the
periodicity, and the number of windows.  The iteration lines have the same
meaning.  When the current value for the PMF is dumped, the format looks
like 

\begin{verbatim}
-172.500000     -172.500000     1.968750        15.394489
-172.500000     -157.500000     2.574512        5.522757
-172.500000     -142.500000     3.147538        2.094142
-172.500000     -127.500000     3.505869        1.141952
\end{verbatim}
where the first two columns are the values of the first and second
dimensions of the reaction coordinate, the third column is the PMF, and the 
last column is the unnormalized probability.  

Once the calculation has converged, wham will produce output resembling

\begin{verbatim}
# Dumping simulation biases, in the metadata file order 
# Window  F (free energy units)
#0       -0.000004
#1       -0.156869
#2       -0.534845
#3       -2.445469
\end{verbatim}

These are the final F values from the wham calculation, and can be used for
computing weighted averages for properties other than the free energy.

If you specified a nonzero number of Monte Carlo bootstrap error analysis
trials, you will see lines that resemble

\begin{verbatim}
#MC trial 0: 990 iterations
#MC trial 1: 973 iterations
#MC trial 2: 970 iterations
#MC trial 3: 981 iterations
#MC trial 4: 984 iterations
\end{verbatim}
at the end of the file.

The free energy data file is written when the calculation converges, and
resembles:

\begin{footnotesize}
\begin{verbatim}
-232.500000     -232.500000     4.812986        0.003185        0.000001        0.000000
-232.500000     -217.500000     4.830312        0.003741        0.000001        0.000000
-232.500000     -202.500000     4.898622        0.001009        0.000000        0.000000
\end{verbatim}
\end{footnotesize}


The first two columns are the locations along the first and second
dimensions of the reaction coordinate.  The third is the free energy, while
the fourth is the statistical uncertainty in the free energy.  The fifth
and sixth columns are the normalized probability and its statistical
uncertainty.  The two uncertainty columns will be zero if you did not use
Monte Carlo bootstrapping.

\section{Discussion}

\subsection{Periodicity}

Use of periodic boundary conditions only changes one thing in the
code: when calculating the biasing potential from a simulation window for a
specific bin in the histogram (the $w_j(X_i)$ values in Equation 8 of Roux's
paper, cited above), the minimum image convention is applied.  Thus, for a
window with the biasing potential centered at 175 degrees, the ``distance''
to the bin at -175 is 10 degrees, not 350 degrees.

The numpad argument on the command line is useful primarily for periodic
reaction coordinates.  It specifies a number of additional windows to be
prepended and appended to the final output, such that the periodicity is
explicitly visible in the free energy.  So, if a calculation was done using
360 degree periodicity, 36 windows, with the reaction coordinate ranging
-180 to 180, and numpad=5, a total of 46 values would be output, from -225
to +225.  The numpad value has no effect at all on the values computed for
the PMF and probability.

\subsection{Monte Carlo Bootstrap Error Analysis}
\label{ss:bootstrap}

The premise of bootstrapping error analysis is fairly straightforward.  For
a time series containing N points, choose a set of N points at random,
allowing duplication.  Compute the average from this ``fake'' data set.
Repeat this procedure a number of times and compute the standard deviation
of the average of the ``fake'' data sets.  This standard deviation is an
estimate for the statistical uncertainty of the average computed using the
real data.  What this technique really measures is the heterogeneity of the
data set, relative to the number of points present.  For a large enough
number of points, the average value computed using the faked data will be
very close to the value with the real data, with the result that the
standard deviation will be low.  If you have relatively few points, the
deviation will be high.  The technique is quite robust, easy to implement,
and correctly accounts for time correlations in the data.  Numerical
Recipes has a good discussion of the basic logic of this technique.  For a
more detailed discussion, see ``An introduction to the bootstrap'', by Efron
and Tibshirani (Chapman and Hall/CRC, 1994).  Please note: bootstrapping can
only characterize the data you have.  If your data is missing contributions
from important regions of phase space, bootstrapping will not help you figure
this out.  

In principle, the standard bootstrap technique could be applied directly to
WHAM calculations.  One could generate a fake data set for each time
series, perform WHAM iterations, and repeat the calculation many times.
However, this would be inefficient, since it would either involve a)
generating many time series in the file system, or b) storing the time
series in memory.  Neither of these strategies is particularly satisfying,
the former because it involves generating a large number of files and the
latter because it would consume very large amounts of memory.  
My implementation of WHAM is very memory efficient because not only does it
not store the time series, it doesn't even store the whole histogram of
that time series, but rather just the nonzero portion.

However, there is a more efficient alternative.  The principle behind
bootstrapping is that you're trying to establish the various of averages
calculated with N points sampling the true distribution function, using
your current N points of data as an estimate of the true distribution.
The histogram of each time series is precisely that, an estimate of the
probability distribution.  So, all we have to do is pick random numbers
from the distribution defined by that histogram.  Once again, Numerical
Recipes shows us how to do it: we compute the normalized cumulant function,
$c(x)$, generate a random number between 0 and 1 $R$, and solve $c(x) = R$ 
for $x$.  Thus, a single Monte Carlo trial is computed in the following manner:

\begin{enumerate}

\item For each simulation window, use the computed cumulant of the histogram
to generate a new histogram, with the same number of points.

\item Perform WHAM iterations on the set of generated histograms

\item Store the average normalized probability and free energy, and their
squares for each bin in the histogram 
\end{enumerate}

There's a subtlety to how you compute fluctuations in the free energy
estimates, since the potential of mean force is only defined up to a
constant.  I have chosen to align the PMFs by computing them from the
normalized probabilities, which is effectively the same as setting the
Boltzmann-averaged free energies equal.  This is a somewhat arbitrary choice
(for example, one could also set the unweighted averages equal), but it seems
reasonable.  If you want something bulletproof, use the probabilities and
their associated fluctuations, which don't have this problem.

The situation is slightly more complicated when one attempts to apply the
bootstrap procedure in two dimensions, because the cumulant is not uniquely
defined.  My approach is to flatten the two dimensional histogram into a 1
dimensional distribution, and take the cumulant of that.  The rest of the
procedure is the same as in the 1-D case.  {\bf In release 2.0.4, the option
to do 2D bootstrapping has been commented out.  I'm not sure if there's a
programming problem, or implementing the better way of doing the 1D case
simply revealed a deeper problem, but 2D bootstrapping is currently broken.}

There is one major caveat throughout all of this analysis: thus far, we have
assumed that the correlation time in time series is shorter than the snapshot
interval.  To put it another way, we've assumed that all of the data points
are statistically independent.  However, this is unlikely to be the case in a
typical molecular dynamics setting, which means that the sample size used in
the Monte Carlo bootstrapping procedure is too large, which in turn causes
the bootstrapping procedure to underestimate the statistical uncertainty.  

My code deals with this by allowing you to set the correlation time for each
time series used in the analysis, in effect reducing the number of points
used in generating the fake data sets (see section ref{ss:format}).  For
instance, if a time series had 1000 points, and you determined by other means
that the correlation time was 10x the time interval for the time series, then
you would set ``correl time'' to 10, and each fake data set would have 100
points instead of 1000.  If the value is unset or is greater than the number
of data points, then the full number of data points is used.  Please note
that the actual time values in the time series are not used in any way in
this analysis; for purposes of specifying the correlation time, the interval
between consecutive points is always considered to be 1.

The question of how to determine the correlation time is in some sense beyond
the scope of this document.  In principle, one could simply compute the
autocorrelation function for each time series; if the autocorrelation is well
approximated by a single exponential, then 2x the decay time (the time it
takes the autocorrelation to drop to $1/e$) would be a good choice.  If it's
multiexponential, then you'd use the longest time constant.  However, be
careful: you really want to use the longest correlation time sampled in the
trajectory, and the fluctuations of the reaction coordinate may fluctuate
rapidly but still be coupled to slower modes.  

It is important to note that the present version of the code uses the
correlation times only for the error analysis and not for the actual PMF
calculation.  This isn't like to be an issue, as the raw PMFs aren't that
sensitive to the correlation times unless they vary by factors of 10 or more.  


\subsection{Using the code for replica exchange simulations}
\label{ss:repex}

One major application for the ability to combine simulations run at different
temperatures is the analysis of replica exchange simulations, and if the
email I've gotten over the last couple of years is any indication, it's a
pretty common one.  My code can be used for replica exchange, but I should
start by admitting that it wasn't designed with it in mind, and may seem a
bit clumsy.  

First, the metadata file format has changed as of the November, 2007 release
of the code.  If you want to specify temperatures in the metadata file, you
also have to specify the number of Monte Carlo points to use (if you're not
using bootstrapping, you can safely set this to any integer).  See section
\ref{ss:format} for details.

In order to use wham with time series collected at different temperatures,
the first thing to do is to follow the instructions given in section
\ref{ss:format} regarding the format of the metadata and time series files,
while setting the spring constants to 0.  Indeed, for simple circumstances
involving small systems this may be enough for you to make a successful
calculation.  

However, for large systems this simple approach will almost certainly get you
nothing but a bunch of NaNs in your output.  If this happens, the most likely
candidate is either a overflow or underflow in the probability histograms.
The reason is that the temperature-sensitive version of the code increments
the histogram by $\exp(-E/k_B T)$ for each point (as opposed to counting each
point as 1).  Since the potential energies for condensed-phase
molecular dynamics systems using standard force fields are typically of order
-50,000 kcal/mol, the means we'd be taking the exponential of a very large
number, which is a Bad Thing numerically.  

However, in many circumstances one can work around this easily, by shifting
the location of zero energy.  The simplest procedure is to locate this lowest
energy in any of the trajectories, and shift \emph{all} of the energies in
\emph{all} of the trajectories such that the lowest (most negative) value is
now zero.  This will eliminate the overflows, since the largest contribution
from an individual data point will now be 1.  

However, shifting the energies upward can lead to a different set of
problems, where a given simulation appears to have no probability associated
with it, e.g. the sum of $\exp(-E/k_B T)$ for the trajectory underflows and
is effectively zero.  This can occur if the energies in the simulation are
significantly higher than those in the lowest energy trajectories, which is
expected for condensed phase systems at high temperatures. Underflow in
itself isn't a problem, but if that simulation is the only one which
contributes to a bin in the histogram (or more generally if all of the
simulations which sample a given bin have zero overall weight), the result
will be a division by zero causing the probability to be NaN or Inf.  

Solving this problem is sometimes quite simple: reshift the energies by a few
kcal/mol, such that the lowest energy is moderately small instead of zero
(say -5 kcal/mol).  If the problem is just numerical underflow, a small shift
may be sufficient to make the problem numerically well-behaved.  However, if
the relevant portion of the histogram really is unaccessible except at high
temperature, then there may be no way to fix the problem, short of running an
additional umbrella-sampled trajectory.



\end{document}
